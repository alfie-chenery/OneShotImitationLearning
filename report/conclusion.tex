\chapter{Conclusion}
\label{chap:conclusion}

In this paper, we have addressed the challenges inherent in robotic manipulation, particularly the need for robots to generalise tasks across varying environments often necessitating Artificially Intelligent components. Through the development of \socalled{LiteBot}, we have successfully demonstrated that it is possible to achieve robust, One-shot Imitation Learning with significantly reduced computational resources compared to currently existing solutions. This work not only replicates the capabilities of state of the art systems like DINOBot \cite{one-shot-imitation} but also extends accessibility by operating efficiently on standard CPU hardware, bypassing the need for expensive high-end GPU resources.\\

In \refchap{chap:evaluation} we demonstrate that while noise and inaccuracies in keypoint matching can introduce discrepancies between the computed and ideal end effector poses, LiteBot consistently manages to complete the majority of tasks successfully. This robustness is evident in \reftab{tab:algo-results}, highlighting LiteBot's capability to perform effectively under less than ideal conditions.\\

Looking ahead, we propose a number of potential directions to take this project further in \refchap{chap:future}. In particular, it remains an ideological goal of this project to remove the last remaining AI component, the embedding transformer. This would allow us to reduce the memory requirements even further, and prove definitively that AI is not a requirement for One-shot Imitation Learning.\\

In summary, LiteBot represents a significant step forward in making advanced robotic manipulation more accessible and efficient. By balancing high performance with lower hardware requirements, this work lowers the barrier of entry for training and experimenting with a One-shot Imitation Learning system, allowing for use in disciplines where traditional hardware constraints may have previously hindered innovation. As we look to the future, the ideas established in this paper will serve as a foundation for further research and development, broadening the practical application and versatility of robotic systems across academic, commercial, and industrial domains.